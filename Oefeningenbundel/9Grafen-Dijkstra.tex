% !TEX root = B&G_oefeningen.tex

\chapter{Algoritme van Dijkstra}

\section*{Inleiding}
Download het bestand \verb/week9-Grafen-Dijkstra-opgave.zip/ van Toledo. Importeer dit bestand in je IDE.

In deze oefenzitting leer je het algoritme van Dijkstra te implementeren in Java.

De eerste oefening is op papier. Hiermee leer je het algoritme goed begrijpen. Verder kan je deze papieren versie gebruiken om je zelfgeschreven code te testen. Houd de oplossing dus goed bij!

\begin{oef}
\label{oef:dijkstra1}
\papier Figuur~\ref{fig:oefDijkstra1} op pagina~\pageref{fig:oefDijkstra1} toont een netwerk. De aangegeven getallen bij
elke verbinding zijn het aantal km tussen beide knooppunten. Stel
een lijst op met de kortste afstanden vanuit \emph{knooppunt 3} naar elk ander
punt. Geef ook telkens de kortste route aan.
\begin{enumerate}
\item Pas het algoritme grafisch toe cfr.\ fig.\ 3.3 in de cursusnota's.
\item Gebruik de tabelmethode (tabel 3.3-3.11 in de cursusnota's).
\end{enumerate}
\begin{figure}[htbp]
     \centering
\input{fig/oef1-Dijkstra.tex}
     \caption{Netwerk bij oefening \ref{oef:dijkstra1}}
     \label{fig:oefDijkstra1}
\end{figure}
\begin{opl}
$\quad$\\
$\quad$\\
\begin{tabular}{ccc}
\toprule
Stad & Kortste afstand vanuit $3$ & Route \\
\midrule
$1$ & 3 & $3\rightarrow 1$ \\
\midrule
$2$ & 4 & $3\rightarrow 2$ \\
\midrule
$3$ & 0 \\
\midrule
$4$ & 9&$3\rightarrow 1 \rightarrow 4$ \\
\midrule
$5$ & 6 & $3\rightarrow 2 \rightarrow 5$\\
\midrule
$6$ & 9 & $3\rightarrow 2 \rightarrow 5 \rightarrow 6$ \\
\midrule
$7$ & 14 & $3 \rightarrow 2 \rightarrow 5 \rightarrow 7$ \\
\midrule
$8$ & 18 & $3 \rightarrow 1 \rightarrow 4 \rightarrow 8$ \\
\bottomrule
\end{tabular}
\end{opl}
\end{oef}


\begin{oef}
\code Je bent eindelijk klaar met het papierwerk. Je kan nu beginnen met het opzetten van de klasse \verb/Graph/.

Schrijf de methode \verb/int[][] initMatrixDijkstra(int vanKnoop)/. Deze methode modelleert de gewichtenmatrix. Let op de indices die je gebruikt!
\begin{itemize}
\item Er wordt een \verb/int[][]/ aangemaakt die evenveel kolommen telt als de gewichtenmatrix, maar één extra rij.
\item Om aan te geven dat de elementen van de laatste rij leeg zijn, stellen we ze gelijk aan \verb/Integer.MAX_VALUE/. 
\item De overige elementen krijgen de corresponderende waarde van de gewichtenmatrix waarbij \verb/infinity/ vervangen wordt door 0.
\item Zet in de kolom die overeenkomt met de startknoop nullen. 
\end{itemize}

Verwachte uitvoer bij het voorbeeld uit de cursustekst, waar een extra knoop $9$ toegevoegd is cfr.\ main methode \verb/UIDijkstra/:
\begin{lstlisting}
Initiele matrix: 

0      	5      	9      	0      	0      	0      	0      	0      	0	
0      	0      	3      	8      	10     	11     	0      	0      	0	
0      	3      	0      	2      	0      	0      	7      	0      	0	
0      	8      	2      	0      	0      	3       7      	0      	0	
0      	10     	0      	0      	0      	1      	0      	8      	0	
0      	0      	0      	3      	1      	0      	5      	10      0	
0      	0      	7      	7      	0      	0      	0      	12     	0	
0      	0      	0       0      	8      	10     	12    	0      	0	
0      	0      	0      	0      	0      	0      	0      	0      	0	
0      	inf    	inf   	inf   	inf   	inf   	inf   	inf   	inf	
\end{lstlisting}
\end{oef}

\begin{oef}
\code Schrijf de methode \verb/int[][] Dijkstra(int vanKnoop)/. Deze methode implementeert het algoritme van Dijkstra. Ze geeft de aangepaste gewichtenmatrix terug, met extra rij onderaan met de kleinste afstanden.

Verwachte uitvoer:
\begin{lstlisting}
Resulterende matrix: 

0	5	0	0	0	0	0	0	0	
0	0	3	0	0	0	0	0	0	
0	0	0	2	0	0	7	0	0	
0	0	0	0	0	3	0	0	0	
0	0	0	0	0	0	0	8	0	
0	0	0	0	1	0	0	0	0	
0	0	0	0	0	0	0	0	0	
0	0	0	0	0	0	0	0	0	
0	0	0	0	0	0	0	0	0	
0	5	8	10	14	13	15	22	inf	
\end{lstlisting}
\begin{opl}
\begin{lstlisting}[caption={Dijkstra}, label=Dijkstra]
private int getAantalKnopen() {
    return gewichtenMatrix.length;
}

private int[][] initMatrixDijkstra(int vanKnoop) {
    int[][] res = new int[this.gewichtenMatrix.length + 1][this.gewichtenMatrix.length];
    // laatste rij is rij met kortste lengtes vanuit vanKnoop

    // oefening 3.3
    for (int i = 0; i < getAantalKnopen(); i++) {
        for (int j = 0; j < getAantalKnopen(); j++)
            res[i][j] = gewichtenMatrix[i][j] != inf ? gewichtenMatrix[i][j] : 0;
            res[getAantalKnopen()][i] = inf;
    }
    for (int i = 0; i <= getAantalKnopen(); i++) {
        res[i][vanKnoop - 1] = 0;
    }
    return res;
}

public int[][] Dijkstra(int vanKnoop) {
    int[][] res = initMatrixDijkstra(vanKnoop);

    System.out.println("Initiele matrix: \n");
    printIntMatrix(res);

    // oefening 3.4
    // herhaal voor alle knopen
    for (int i = 0; i < getAantalKnopen() - 1; i++) {
        // zoek nieuwe minimale afstand
        int min = inf;
        int[] knopenpaar = {inf, inf}; // index die het nieuwe minimum is
        for (int j = 0; j < getAantalKnopen(); j++) {
            // herhaal voor alle knopen die al bezocht zijn
            if (res[getAantalKnopen()][j] != inf) {
                for (int k = 0; k < getAantalKnopen(); k++) {
                    // als knoop k+1 nog niet gevonden is,
                    // als er een verbinding is tussen knoop j+1 en knoop k+1
                    // en als de verbinding tussen deze knopen korter is 
                    // dan het minimum tot nog toe
                    if (res[getAantalKnopen()][k] == inf && res[j][k] != 0 && 
                      res[getAantalKnopen()][j] + res[j][k] < min) {
                        // onthoud (index van) dit knopenpaar en hun minimum
                        knopenpaar[0] = j;
                        knopenpaar[1] = k;
                        min = res[getAantalKnopen()][j] + res[j][k];
                    }
                }
            }
        }
        // tussenresultaat wegschrijven indien er verbetering is
        if (knopenpaar[0] != inf && knopenpaar[1] != inf) {
            // nieuwe minimum
            res[getAantalKnopen()][knopenpaar[1]] = min;
            for (int j = 0; j < getAantalKnopen() - 1; j++) {
                // kolom op nul zetten, maar niet op de plaats die het minimum aanlevert
                if (j != knopenpaar[0])
                    res[j][knopenpaar[1]] = 0;
            }
        }
    }
    return res;
}
\end{lstlisting}
\end{opl}

\end{oef}

\begin{oef}
\code Schrijf \verb/ArrayList<Integer> vindPad(int vanKnoop, int naarKnoop, int[][] res)/, een methode die het kortste pad van \verb/vanKnoop/ naar \verb/naarKnoop/ berekent. De veranderlijke \verb/int[][] res/ is de aangepaste gewichtenmatrix zoals die teruggegeven wordt door de methode \verb/Dijkstra/.

Verwachte uitvoer:
\begin{lstlisting}
Kortste afstand van 1 naar 2 = 5
via [1, 2]
Kortste afstand van 1 naar 3 = 8
via [1, 2, 3]
Kortste afstand van 1 naar 4 = 10
via [1, 2, 3, 4]
Kortste afstand van 1 naar 5 = 14
via [1, 2, 3, 4, 6, 5]
Kortste afstand van 1 naar 6 = 13
via [1, 2, 3, 4, 6]
Kortste afstand van 1 naar 7 = 15
via [1, 2, 3, 7]
Kortste afstand van 1 naar 8 = 22
via [1, 2, 3, 4, 6, 5, 8]
Er is geen pad van 1 naar 9
\end{lstlisting}
\begin{opl}
\begin{lstlisting}[caption=vindPad, label=DijkstravindPad]
private ArrayList<Integer> vindPad(int vanKnoop, int naarKnoop, int[][] res) {
    ArrayList<Integer> pad = new ArrayList<>();
    // oefening 3.5
    // naarKnoop, vanKnoop en k zijn namen van knopen
    // hun index in de matrix is altijd eentje minder want de rijen/kolommen tellen vanaf 0
    pad.add(naarKnoop);

    while (naarKnoop != vanKnoop) {
        int k = 1;
        while (k - 1 < getAantalKnopen() && res[k - 1][naarKnoop - 1] == 0)
            k++;
            pad.add(0, k);
            naarKnoop = k;
        }
    return pad;
}
\end{lstlisting}
\end{opl}

\end{oef}




